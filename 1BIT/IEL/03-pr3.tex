\section{Příklad 3}
% Jako parametr zadejte skupinu (A-H)
\tretiZadani{C}

\subsection{Nahrazení napěťového zdroje za proudový}
Pro náhradu použijme I\textsubscript{n} = G\textsubscript{1}U (pro snazší počítání vodivosti)
%to[R, l^=R\textsubscript{1}]
%node[circ]{} (2,4)
%to[dcvsource, v^<=U]
% to[ioosource, i=\I{n}]
\begin{figure}[H]
    \centering
    \begin{circuitikz}
    \draw (-1,0) to[ioosource, i=I\textsubscript{n}] (-1,2)
    (-1,2) -- (2,2)
    (0,2) node[circ]{} (0,2)
    (0,2) to[R, l^=G\textsubscript{1}] (0,0)
    (0,0) node[circ]{} (0,0)
    (2,2) node[circ]{} (2,2)
    (2,2) to[R, l^=G\textsubscript{2}] (4,2)
    (4,2) node[circ]{} (4,2)
    (4,2) to[R, l^=G\textsubscript{4}] (4,0)
    (4,0) node[circ]{} (4,0)
    -- (5,0)
    (4,2) -- (5,2)
    (5,2) to[ioosource, i=I\textsubscript{2}] (5,0)
    (4,0) -- (4,-1)
    (4,-1) to[ioosource, i=I\textsubscript{1}] (2,-1)
    -- (2,0)
    (2,0) node[circ]{} (2,0)
    -- (-1,0)
    (2,0) to[R, l^=G\textsubscript{5}] (4,0)
    (2,0) to[R, l^=G\textsubscript{3}] (2,2)
    ;
    \end{circuitikz}
\end{figure}

\subsection{Výpočet uzlových napětí}
Použijme metodu uzlových napětí a vypočítáme U\textsubscript{A}, U\textsubscript{B}, U\textsubscript{C}

\begin{figure}[H]
    \centering
    \begin{circuitikz}
    \draw (-1,0) to[ioosource, i=I\textsubscript{n}] (-1,2)
    (-1,2) -- (2,2)
    (0,2) node[circ]{} (0,2)
    (0,2) to[R, l^=G\textsubscript{1}] (0,0)
    (0,0) node[circ]{} (0,0)
    (2,2) node[circ]{} (2,2)
    (2,2) to[R, l^=G\textsubscript{2}] (4,2)
    (4,2) node[circ]{} (4,2)
    (4,2) to[R, l^=G\textsubscript{4}] (4,0)
    (4,0) node[circ]{} (4,0)
    -- (5,0)
    (4,2) -- (5,2)
    (5,2) to[ioosource, i=I\textsubscript{2}] (5,0)
    (4,0) -- (4,-1)
    (4,-1) to[ioosource, i=I\textsubscript{1}] (2,-1)
    -- (2,0)
    (2,0) node[circ]{} (2,0)
    -- (-1,0)
    (2,0) to[R, l^=G\textsubscript{5}] (4,0)
    (2,0) to[R, l^=G\textsubscript{3}] (2,2)
    {[anchor=south] (2,2) node {A} (4,2) node {B}}
    {[anchor=north west] (4,0) node {C}}
    ;
    \end{circuitikz}
\end{figure}

\begin{align*}
    I\textsubscript{R1} - I\textsubscript{R2} - I\textsubscript{R3} = 0 \\
    I\textsubscript{R2} - I\textsubscript{R4} - I\textsubscript{2} = 0\\
    I\textsubscript{R4} + I\textsubscript{2} - I\textsubscript{R5} - I\textsubscript{1} = 0
\end{align*}

\begin{align*}
    G\textsubscript{1}(U - U\textsubscript{A}) - G\textsubscript{2}(U\textsubscript{A} - U\textsubscript{B} - U\textsubscript{C}) - G\textsubscript{3}U\textsubscript{A} = 0 \\
    G\textsubscript{2}(U\textsubscript{A} - U\textsubscript{B} - U\textsubscript{C}) - G\textsubscript{4}(U\textsubscript{B} - U\textsubscript{C}) = I\textsubscript{2} \\
    G\textsubscript{4}(U\textsubscript{B} - U\textsubscript{C}) - G\textsubscript{5}U\textsubscript{C} = I\textsubscript{1} - I\textsubscript{2}
\end{align*}

\begin{align*}
    (G\textsubscript{1} + G\textsubscript{2} + G\textsubscript{3})U\textsubscript{A} - G\textsubscript{2}U\textsubscript{B} - G\textsubscript{2}U\textsubscript{C} = G\textsubscript{1}U \\
    G\textsubscript{2}U\textsubscript{A} - (G\textsubscript{2} + G\textsubscript{4})U\textsubscript{B} - (G\textsubscript{2} - G\textsubscript{4})U\textsubscript{C} = I\textsubscript{2} \\
    G\textsubscript{4}U\textsubscript{B} - (G\textsubscript{4} + G\textsubscript{5})U\textsubscript{C} = I\textsubscript{1} - I\textsubscript{2}
\end{align*}    

Teď převedeme soustavu na matice (A je matice vodivosti, x je matice napětí, B je matice proudu). A pak najdeme matici x pomocí $Ax = B$ => $x = A^{-1}B$ (To už bude při dosazováni)

\begin{figure}[H]
\begin{align*}
\begin{pmatrix}
	G_1+G_2+G_3&-G_2&-G_2\\
	G_2&-G_2-G_4&G_4-G_2\\
	0&G_4&-G_4-G_5
\end{pmatrix}\times 
\begin{pmatrix}
	U_A\\
	U_B\\
	U_C
\end{pmatrix}=
\begin{pmatrix}
	G_1U\\
	I_2\\
	I_1-I_2
\end{pmatrix}
\end{align*}
\end{figure}

\subsection{Výpočet U\textsubscript{R4} a I\textsubscript{R4}}
Nyní zbývá jen vypočítat napětí na odporu R\textsubscript{4} a pak už stačí jen použít Ohmův zákon k určení proudu.

\begin{align*}
    U\textsubscript{R4} = U\textsubscript{B} - U\textsubscript{C} \\
    I\textsubscript{R4} = \frac{U\textsubscript{R4}}{R\textsubscript{4}}
\end{align*}

\subsection{Dosazení}
\begin{figure}[H]
\begin{align*}
\begin{pmatrix}
	0.0728&-0.03223&-0.0323\\
	0.0323&-0.08223&0.0177\\
	0&0.05&-0.3833
\end{pmatrix}\times 
\begin{pmatrix}
	U_A\\
	U_B\\
	U_C
\end{pmatrix}=
\begin{pmatrix}
	2.5\\
	0.75\\
	0.1
\end{pmatrix}
\end{align*}
\begin{align*}
\begin{pmatrix}
	U_A\\
	U_B\\
	U_C
\end{pmatrix}=
\begin{pmatrix}
	19.97&-14.8259&-10.1943\\
	9.5829&-20.4602&-7.8726\\
	5.7437&-12.3361&-16.1216
\end{pmatrix}\times
\begin{pmatrix}
	2.5\\
	0.75\\
	0.1
\end{pmatrix}
\end{align*}
\end{figure}

\begin{align*}
    U\textsubscript{A} = \SI{37.786}{\volt} \\
    U\textsubscript{B} = \SI{7.8248}{\volt} \\
    U\textsubscript{C} = \SI{3.495}{\volt}
\end{align*}

\begin{align*}
    U\textsubscript{R4} = U\textsubscript{B} - U\textsubscript{C} = 7.8248 - 3.495 = \SI{4.3298}{\volt} \\
    U\textsubscript{R4} = \frac{U\textsubscript{R4}}{R\textsubscript{4}} = \frac{4.3298}{20} = \SI{216.4859}{\milli\ampere}
\end{align*}