\documentclass[a4paper,11pt,twocolumn]{article}

\usepackage{geometry}
\geometry{a4paper, total={192mm,260mm}, centering}

\usepackage[T1]{fontenc}
\usepackage[utf8]{inputenc}
\usepackage{lmodern}
\usepackage{hyperref}

\usepackage{amsfonts}
\usepackage{amsmath}
\usepackage{amsthm} 

\newtheorem{definition}{Definice}
\newtheorem{veta}{Věta}

\begin{document}

\thispagestyle{empty}
\onecolumn
\begin{center}
    {\Huge\textsc{Vysoké učení technické v Brně}}\\[18pt]
    {\huge\textsc{Fakulta informačních technologií}}\\[250pt]
    
    {\huge Typografie a publikování -– 2. projekt}\\[5pt]
    {\huge Sazba dokumentů a matematických výrazů}
\end{center}

\vfill

\makebox[\textwidth]{
        \Large 2025
        \hfill
        \Large Smirnov Nikita (xsmirn02)
    }
\clearpage

\setcounter{page}{1}
\twocolumn

\section*{Úvod}
V této úloze vysázíme titulní stranu a ukázku matematického textu,
v němž se vyskytují například
rovnice \eqref{eq:7} na straně \pageref{eq:1}, Věta \ref{v:1} nebo Definice \ref{def:2}.
Pro vytvoření těchto odkazů používáme kombinace příkazů
\verb|\label|, \verb|\ref|, \verb|\eqref| a \verb|\pageref|.
Před odkazy patří nezlomitelná mezera.
Text zvýrazníme pomocí příkazu \verb|\emph|, strojopisné písmo pomocí \verb|\texttt|.
Pro \LaTeX ové příkazy (s obráceným lomítkem) použijeme \verb|\verb|.

Titulní strana je vysázena prostředím \texttt{titlepage} a nadpis je v optickém středu
s využitím \textit{zlatého řezu}, který byl probrán na přednášce.
Na titulní straně jsou tři různé velikosti písma a mezi dvojicemi řádků textu
je řádkování se zadanou  velikostí 0,5\,em a 0,6\,em\footnote{Použijte správnou velikost mezery mezi číslem a jednotkou.}.

\section{Matematický text}
Symboly číselných množin sázíme makrem \verb|\mathbb|,
kaligrafická písmena  makrem \verb|\mathcal|.
Pozor na tvar i sklon řeckých písmen: srovnejte \verb|\rho| a \verb|\varrho|.
Konstrukce \verb|${}$| nebo \verb|\mbox{}| zabrání zalomení výrazu.

Pro definice a věty slouží prostředí definovaná příkazem \verb|\newtheorem| z balíku \texttt{amsthm}.
Tato prostředí obracejí význam \verb|\emph|:
uvnitř textu sázeného kurzívou se zvýrazňuje písmem v základním řezu.
Důkazy se někdy ukončují značkou \verb|\qed|.

\subsection{Pseudometrický prostor}
Pro zarovnání rovností a nerovností pod sebe použijte vhodné prostředí.

\begin{definition}
V \emph{pseudometrickém prostoru} $\mathcal{M} = (M, \varrho)$ značí $M$ množinu bodů,
$\varrho : M \times M \rightarrow \mathbb{R}$ je zobrazení zvané \emph{pseudometrika}, které pro každé body $x, y, z \in M$
splňuje následující podmínky:

\begin{align}
    \varrho(x,x) &= 0 \label{eq:1} \\
    \varrho(x,y) &= \varrho(y,x) \label{eq:2} \\
    \varrho(x,y) + \varrho(y,z) &\geq \varrho(x,z) \label{eq:3}
\end{align}
\end{definition}

\subsection{Metrika}
Funkční hodnota pseudometriky $\varrho$ se nazývá \textit{vzdálenost}.
Vzdálenost každých dvou bodů je nezáporná.

\begin{veta} \label{v:1}
Pro každé dva body $x,y \in M$ pseudometrického prostoru $(M,\varrho)$ platí $\varrho(x,y) \geq 0$.
\end{veta}

Důkaz: Nechť $x, y \in M$ a označme $d = \varrho(x,y)$. Využitím \eqref{eq:2} máme $2d = \varrho(x,y) + \varrho(y,x)$, z nerovnosti \eqref{eq:3} vyplývá $2d \geq \varrho(x,x)$ a z rovnosti \eqref{eq:1} dostaneme $2d \geq \varrho(x,x) = 0$. Odtud plyne $d \geq 0$. \qed

Speciálním případem pseudometrických prostorů jsou prostory metrické,
v nichž dva různé body mají vždy kladnou vzdálenost.

\begin{definition}
\label{def:2}
Nechť $\mathcal{M} = (M,\varrho)$ je pseudometrický prostor, v němž platí $\varrho(x,y) > 0$ kdykoliv $x \ne y$.
Potom $\mathcal{M}$ se nazývá \emph{metrický prostor}
a $\varrho$ je jeho \emph{metrika}.
\end{definition}

\section{Rovnice}
Velikost závorek a svislých čar je potřeba přizpůsobit jejich obsahu.
K tomu jsou určeny modifikátory \verb|\left| a \verb|\right|.
\begin{align}
    \lim_{p \rightarrow 0}
    \left( 
    \frac{1}{n}\sum^n_{i=1}{x^p_i} 
    \right)^\frac{1}{p}
    = 
    \left(
    \prod_{i=1}^n{x_i}
    \right)^\frac{1}{n}
\end{align}

Zde vidíme, jak se vysází proměnná určující limitu v běžném textu: $\lim_{m\rightarrow\infty}f(m)$.
Podobně je to i s dalšími symboly jako $\bigcup_{N\in\mathcal{M}}N$ či $\sum^{m}_{i=1}x^2_i$.
S vynucením méně úsporné sazby příkazem \verb|\limits| budou vzorce vysázeny v podobě $
\lim_{m\rightarrow\infty}\limits f(m)$ a $\sum^{m}_{i=1}\limits x^2_i$.
Složitější matematické formule sázíme mimo plynulý text pomocí prostředí \texttt{displaymath}.

\begin{align}
    \lim_{n\rightarrow\infty}\left(1+\frac{x}{n}\right)^n &= \sum^{\infty}_{n=0}\frac{x^n}{n!} \\
    \sum_{\emptyset \ne X \subseteq P}(-1)^{|X|-1} \left|\bigcap X\right| &= \left| \bigcup P \right| \\ 
    - \int^{b}_{a}f(x)\mathrm{d}x &= \int^a_b f(y)\mathrm{d}x \label{eq:7}
\end{align}
Nezapomeňte rovnice, na které se odkazujete, označit vhodným jménem pomocí \verb|\label|.

\section{Matice}
Pro sázení matic se používá prostředí \texttt{array} a závorky s výškou nastavenou pomocí
\verb|\left|, \verb|\right|.

$D = \left| \begin{array}{cccc}
     a_{11} & a_{12} & \dots & a_{1n} \\
     a_{21} & a_{22} & \dots & a_{2n} \\
     \vdots & \vdots & \ddots & \vdots \\
     a_{m1} & a_{m2} & \dots & a_{mn} 
\end{array} \right| 
= 
\left|
\begin{array}{cc}
     x & y \\
     t & w
\end{array}
\right|
=
xw-yt
$
Prostředí \verb|array| lze úspěšně využít i jinde,
například na pravé straně následující definiční rovnosti.
\[
B_n =
\left\{
\begin{array}{ll}
1 & \text{pro } n = 0 \\[10pt]
\sum\limits_{k=0}^{n-1} \binom{n}{k} B_k & \text{pro } n \geq 1
\end{array}
\right.
\]

Jestliže sázíme jen levou složenou závorku, pak za párovým \verb|\right|
místo závorky píšeme tečku.


\end{document}
