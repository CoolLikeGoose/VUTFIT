\documentclass[a4paper,11pt]{article}

\usepackage[left=2cm, top=3cm, text={17cm, 24cm}]{geometry}
\usepackage[czech]{babel}
\usepackage[utf8]{inputenc}
\usepackage{times}
\usepackage{hyperref}
\usepackage{cite}
\usepackage{csquotes}

% \usepackage[backend=bibtex,style=numeric]{biblatex}

\begin{document}

\thispagestyle{empty}
\begin{center}
    \Huge \textsc{Vysoké učení technické v Brně}\\
    \huge\textsc{Fakulta informačních technologií}\\
    
    \vspace{\stretch{0.382}}
    
    Typografie a publikování -- 4. projekt\\
    \textbf{Bibliografické citace}\\
    
    \vspace{\stretch{0.618}}
\end{center}

{\LARGE 24. dubna 2025 \hfill Smirnov Nikita (xsmirn02)}
\clearpage

\setcounter{page}{1}

\section{Úvod}
V posledních letech kvantové počítače zaznamenaly výrazný pokrok a staly se jedním z nejdiskutovanějších témat v oblasti výpočetní techniky. Díky investicím velkých technologických společností, jako je Microsoft a IBM, se vývoj kvantových technologií rychle posouvá kupředu \cite{novinky2025qcvyvoj}.

\section{Jak kvantové počítače fungují}
Na rozdíl od klasických počítačů, které používají bity s hodnotami 0 nebo 1, kvantové počítače využívají \emph{qubity}, které mohou být současně v superpozici stavů 0 a 1. Tato vlastnost umožňuje paralelní zpracování informací a potenciálně exponenciální zrychlení některých výpočtů \cite{nielsen2000quantum}.

Pro stabilní provoz qubitů je nutné dosáhnout extrémně nízkých teplot blízkých absolutní nule. Například supravodivé qubity vyžadují chlazení na teploty kolem 15 milikelvinů, aby se minimalizoval šum a dekoherence \cite{gruska2003quantum}.

\section{Dopad kvantových počítačů}

\subsection{Negativní aspekty}
Jedním z hlavních obav spojených s kvantovými počítači je jejich schopnost prolomit současné kryptografické algoritmy, jako je RSA. Shorův algoritmus\cite{shor1994algo} umožňuje efektivní faktorizaci velkých čísel, což by mohlo ohrozit bezpečnost digitální komunikace.

\subsection{Pozitivní přínosy}
Na druhé straně kvantové počítače nabízejí obrovský potenciál v oblasti simulace přírodních procesů\cite{james2011sim}. Mohou například modelovat komplexní chemické reakce, což by mohlo urychlit vývoj nových léků a materiálů \cite{De_Wolf2017-ku}.

\section{Současný vývoj a výzvy}
Zvyšování počtu qubitů v kvantových počítačích je technicky náročné. Microsoft nedávno představil čip Majorana 1, který využívá nový stav hmoty zvaný topokonduktor. Tato technologie by mohla umožnit škálování kvantových počítačů na miliony qubitů \cite{microsoft2025break}.

Kromě toho se zkoumají možnosti propojení více kvantových počítačů do distribuovaných systémů, což by mohlo překonat omezení jednotlivých zařízení \cite{carter2024qcUni}.

\section{Závěr}
Kvantové počítače představují revoluční technologii s potenciálem transformovat mnoho oblastí lidské činnosti. Ačkoli čelí mnoha výzvám, pokrok v této oblasti je slibný. Vývoj kvantových počítačů také otevírá nové odvětví programování, protože psaní algoritmů pro kvantové počítače se velmi liší od obvyklého přístupu v programování\cite{algo1999mosca}. Osobně jsem začal studovat kvantové programování prostřednictvím přístupného článku na Habr \cite{habr2020}.

\bibliographystyle{czechiso}
\bibliography{refs}
\end{document}
